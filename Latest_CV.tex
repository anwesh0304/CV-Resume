\documentclass[a4paper]{article}
    \usepackage{fullpage}
    \usepackage{amsmath}
    \usepackage{fontawesome}
    \usepackage{bbding}
    \usepackage{amssymb}
    \usepackage{marvosym}
    \usepackage{textcomp}
    \usepackage[utf8]{inputenc}
    \usepackage[T1]{fontenc}
    \usepackage[margin=1in]{geometry}
    \textheight=10in
    \pagestyle{empty}
    \raggedright

    %\renewcommand{\encodingdefault}{cg}
%\renewcommand{\rmdefault}{lgrcmr}

\def\bull{\vrule height 0.8ex width .7ex depth -.1ex }

% DEFINITIONS FOR RESUME %%%%%%%%%%%%%%%%%%%%%%%

\newcommand{\area} [2] {
    \vspace*{-9pt}
    \begin{verse}
        \textbf{#1}   #2
    \end{verse}
}

\newcommand{\lineunder} {
    \vspace*{-8pt} \\
    \hspace*{-18pt} \hrulefill \\
}

\newcommand{\header} [1] {
    {\hspace*{-18pt}\vspace*{6pt} \textsc{#1}}
    \vspace*{-6pt} \lineunder
}

\newcommand{\employer} [3] {
    { \textbf{#1} (#2)\\ \underline{\textbf{\emph{#3}}}\\  }
}

\newcommand{\contact} [3] {
    \vspace*{-10pt}
    \begin{center}
        {\Huge \scshape {#1}}\\
        #2 \\ #3
    \end{center}
    \vspace*{-8pt}
}

\newenvironment{achievements}{
    \begin{list}
        {$\bullet$}{\topsep 0pt \itemsep -2pt}}{\vspace*{4pt}
    \end{list}
}

\newcommand{\schoolwithcourses} [4] {
    \textbf{#1} #2 $\bullet$ #3\\
    #4 \\
    \vspace*{5pt}
}

\newcommand{\school} [4] {
    \textbf{#1} #2 $\bullet$ #3\\
    #4 \\
}

\newcommand{\combinesum}[3]{ \displaystyle{\sum_{r=0}^{#1} {r^{#3}} {#1 \choose #2}}} 

% END RESUME DEFINITIONS %%%%%%%%%%%%%%%%%%%%%%%

    \begin{document}
\vspace*{-40pt}

%==== Profile ====%
\vspace*{-10pt}
\begin{center}
	{\Huge \scshape {Anwesh Bhattacharya}}\\
	Gandhi Bhawan - 3131, BITS-Pilani, Pilani, Rajasthan - 333031, India \\ \Email \ f2016590@pilani.bits-pilani.ac.in \ \PhoneHandset +919116702059\\
	\ifthenelse{\equal{\cvgithub}{}}{}{\textsc{\Large\faGithub}} \cvgithub anwesh0304 \\
	
\end{center}

%==== Education ====%
\header{Education}
\textbf{Birla Institute of Technology and Science}\hfill Pilani, Rajasthan\\
M.Sc (Hons) Phyiscs + B.E. (Hons) Computer Science \\
\textbf{CGPA} : 9.14 \hfill August 2016 - Present\\
\vspace{2mm}
\textbf{National Public School, Koramangala}\hfill Bangalore, Karnataka\\
Elective : Computer Science. \hfill June 2014 - April 2016\\
\textbf{Percentage} : 97.2\% \\
\vspace{2mm}
\textbf{Baldwin Boys High School}\hfill Bangalore, Karnataka\\
Elective : Computer Applications. \hfill June 2005 - April 2014\\
\textbf{Percentage} : 95.6\% \\
\vspace{2mm}

%==== Experience ====%
\header{Research Experience}
\vspace{1mm}

\textbf{National Institute of Advanced Studies (NIAS)} \hfill Bangalore\\
\textit{Research Intern} \hfill May 2020 - Present\\
\vspace{-1mm}
\begin{itemize} \itemsep 1pt
	\item Improving \textbf{PSO} algorithm with the application of \textbf{chaotic flows/maps} 
	\item Scientifically reasoning the improvement of performance
	\item Approximation of gradients in \textbf{non-differentiable} objective functions
	\item Transferring the technique to chaotic firing of \textbf{neural network} for \textbf{classification problems}
\end{itemize}

\textbf{Indian Institute of Astrophysics (IIA)} \hfill Bangalore\\
\textit{Research Intern} \hfill May 2019 - July 2019\\
\vspace{-1mm}
\begin{itemize} \itemsep 1pt
	\item Worked towards catalouging Double Nuclei Galaxies from SDSS with \textbf{Image Processing} under \textit{Dr. Mousumi Das}. Developing such a catalog is crucial to studying galaxy mergers
	\item Used Python and utilized libraries such as \textbf{Numpy/OpenCV/Astropy/Web Scraping} libraries to process the FITS images of galaxies in the R-band.
	\item Implemented \textbf{Optimization Techniques} (Gradient Ascent) and \textbf{Graph Algorithms} to classify galaxies having single or double nuclei
	\item Obtained an accuracy of \textbf{94\%} on the catalog by \textit{Gimeno} et. al. (2004)
	\item Tested the pipeline for stability and released it on GitHub. Code available at - \texttt{https://github.com/anwesh0304/anwesh-DAGN}. Preprint at \texttt{https://bit.ly/31ifeUP}
\end{itemize}

\textbf{Inter-University Centre for Astronomy and Astrophysics (IUCAA)} \hfill Pune\\
\textit{Research Intern} \hfill May 2018 - July 2018\\
\vspace{-1mm}
\begin{itemize} \itemsep 1pt
	\item Supervised by the Director of IUCAA, \textit{Dr. Somak Raychaudhury}
	\item Worked towards The Detection of Patterns in the Cosmic Web in the COMA Supercluster using \textbf{Mathematical Morphology}
	\item Revamped the DisPerSE source code, which was released in 2011, to run on Ubuntu 16.04 LTS
	\item Fully installed all code dependencies and obtained experience in using the UNIX shell
	\item Identified a set of five clusters, including the \textbf{Abell cluster}, and the connecting filaments in COMA.
\end{itemize}

\header{Publications}
{\textbf{Stirling Numbers Via Combinatorial Sums}}\hfill June 2019\\
\vspace{-1mm}
\begin{itemize} \itemsep 1pt
    \item Analysed summations of the type $\combinesum{n}{r}{k}$ for general $k$. 
    \item Obtained the recurrence for the Stirling Numbers of the First and Second Kind in a \textbf{novel Manner}
    \item Verified the results with Online Encylcopedia of Integer Sequences (OEIS)
    \item Presented at the \textbf{International Conference on Modelling, Machine Learning and Astronomy} 2019, at PES University, Bangalore.
    \item Preprint available at \texttt{http://bit.ly/2k951dF}
\end{itemize}
\vspace*{2mm}

\header{Academic Projects}
{\textbf{ERPLAG Compiler (BITS-Pilani)}} \hfill January 2020 - May 2020\\
\begin{itemize} \itemsep 1pt
    \item Created a 64-bit compiler for a toy language \textbf{ERPLAG} in \textbf{C} without the help of any additional libraries
    \item Implemented features such as \textbf{expressions}, \textbf{dynamic array abstraction} and \textbf{multi-return function calls}
    \item Tested rigorously for \textbf{portability} on various Linux distributions and Windows.
\end{itemize}
\vspace*{2mm}

{\textbf{Machine Learning on FPGA (CEERI)}}\hfill January 2020 - May 2020\\
\begin{itemize} \itemsep 1pt
    \item Learning to use High Level Synthesis (\textbf{HLS}) in \textbf{C++} for synthesis of accelerators
    \item Designing a simple classifier on hardware to perform handwritten digit recognition from \textbf{MNIST}.
\end{itemize}
\vspace*{2mm}

{\textbf{Special Topics in Quantum Mechanics (BITS-Pilani)}}\hfill August 2019 - December 2019\\
\begin{itemize} \itemsep 1pt
    \item Studying the historical aspects and subtle topics of Quantum Mechanics which are not taught in detail in an undergraduate course
    \item Read the work by \textit{Tomonaga} on the foundations of \textbf{blackbody radiation, Planck's hypothesis} and \textbf{Einstein's corpuscular theory}
    \item Studied topics such as \textbf{EPR Paradox, Bell's Inequality}.
    \item Exposed to advanced topics such as \textbf{Feynman's Path Integral Formulation, Hamilton-Jacobi theory.}
\end{itemize}
\vspace*{2mm}


{\textbf{Dark Energy Modelling and Gravitational Lensing (BITS-Pilani)}}\hfill August 2019 - December 2019\\
\begin{itemize} \itemsep 1pt
    \item Studying the \textbf{FLRW} metric and background cosmology to model the equation of state for dark energy.
    \item  Used the \textbf{7-CPL model} to obtain Hubble parameter, luminosity and angular-diameter distances as a function of redshift.
    \item Used the available code of Eisenstein et. al. to obtain growing mode and power spectrum
\end{itemize}
\vspace*{2mm}

\header{Courses}
\begin{itemize} \itemsep 1pt
	\item Math: Linear Algebra, Differential Equations, Numerical Techniques    
	\item Physics: Classical Mechanics, Electromagnetic Theory, Quantum Computing, Statistical Physics, General Relativity, Solid-State Physics, Nuclear Physics
	\item Computer Science: Data Structures \& Algorithms, OOP, Database Systems, Operating Systems, Computer Architecture, Theory of Computation, Compiler Construction, Computer Networks, Parallel Computing
	\item Coursera (Completed): 
	\begin{itemize}
	    \item Machine Learning (Certificate : 6WSURAQVC6PF)
	    \item Tensorflow Specialisation - I (Certificate : LUFURSD8ABEK)
	    \item Deep Learning Specialisation - I (Certificate : WLJ3EQ3Z5BPD)
	    \item Deep Learning Specialisation - III (Certificate : JF27NLTEQAXP)
	\end{itemize}
\end{itemize}
\vspace{2mm}

\newpage

\header{Skills}
\begin{itemize}
	\item Programming Langauges
	\begin{itemize}
	    \item Proficient: C, Python, MATLAB, \LaTeX
	    \item Intermediate : C++, Java, Verilog, bash
	    \item Beginner : Haskell, Scheme, batch
	\end{itemize}
	\item Modules: Numpy, Tensorflow, Keras, Astropy, OpenCV, BeautifulSoup, STL
	\item Version Control : git
\end{itemize}
\vspace{2mm}

\header{Awards}
\begin{itemize}
    \item \textbf{Innovation in Science Pursuit for Inspired Research (INSPIRE) Scholarship} \\ \textbf{Department of Science and Technology (DST)}\\ \vspace{2mm}
Awarded the scholarship for excellent performance in AISSCE (CBSE 12th) board examinations and for securing an all-India rank of 1100 in JEE Mains 2016\\

    \item \textbf{Prabhat Award for Best Outgoing Student in Physics}  \\ \textbf{Department of Physics (BITS-Pilani)}\\ \vspace{2mm}
Ascertained as the best student of the batch of Physics 2016 with respect to academic performance, projects and future research plans
\end{itemize}


\header{Extra-curricular Activities}
\begin{itemize}
    \item Music : I play guitar, keyboard, drums and I'm interested in music production
    \item Animal Welfare : I raised Rs 11,000 for an injured dog in my college dorm
    \item Gymming : I take a keen interest in body-building
\end{itemize}
\vspace*{2mm}

\header{Languages}
English, Hindi and Bengali
\vspace*{2mm}

\ 
\end{document}
